%                 Doubly-Stopped Binomial Distribution
%                        Kane and Zelterman
\typeout{}\typeout{}\typeout{}\typeout{}
\typeout{ Two-Stage, Phase II Designs }
\typeout{}\typeout{}
\typeout{ Michelle DeVeaux \& Michael Kane \& Dan Zelterman }
\typeout{}\typeout{}\typeout{}\typeout{}


\documentclass[12pt]{article}         %LaTeX preamble
\parskip   .4em
\jot  2ex
\parindent .3in
\textwidth      6.5in
\oddsidemargin    0in
\textheight 8.75in
\topmargin      -.5in
\baselineskip    26pt
\renewcommand{\baselinestretch}{1.5}
\renewcommand{\arraystretch}{.75}     % vertical spacing in tables
\usepackage{graphicx}                 % Needed to import pdf files
\usepackage{adjustbox}
\usepackage{amsmath}
\usepackage{amsfonts}
\usepackage{amsthm}
\usepackage{url}
\DeclareMathOperator*{\argmin}{argmin}
\newcommand*{\argminl}{\argmin\limits}
\begin{document}

\newtheorem{prop}{Proposition}


% --------------- New command definitions ----------------------

\newcounter{bibno}%            Bibliographic counter

\def\waux{1}%                  Write number for the .AUX file

\newcommand\writelabel[2]%     Write label[#1] and value[#2] to .aux file
   {%                          General command, called by \Aexer and \Texer
     \immediate\write\waux%         Write this information to the .aux file:
          {\noexpand\newlabel{#1}%  Arguments #1, #2, and the page number
             {{#2}{\thepage}}}%     Warning: the page number will be incorrect
   }%                               because I use the \immediate command

\newcommand\bibref[1]%              Reference name and my label (#1)
   {%
    \addtocounter{bibno}{1}%        Increment the counter
    \def\biblab{\thebibno}%         name is same as counter with dot
    \writelabel{#1}{\biblab}%       Write the label to .AUX file 
    \noindent\biblab.%              Write this exercise name for \item[ ]
   }%

% -------------------  global definitions  ------------------------

\newcommand{\0}{\mbox{\boldmath $0$}}         %  bold zero in math mode
\newcommand{\bbeta}{\mbox{\boldmath $\beta$}} %  bold beta in math mode
\newcommand{\bd}{\mbox{\boldmath $d$}}        %  bold d in math mode
\newcommand{\bg}{\mbox{\boldmath $\gamma$}}   %  bold gamma in math mode
\newcommand{\blam}{\mbox{\boldmath $\lambda$}}%  bold lambda in math mode
\newcommand{\bpi}{\mbox{\boldmath $\pi$}}     %  bold pi in math mode
\newcommand{\bsig}{\mbox{\boldmath $\sigma$}} %  bold sigma in math mode
\newcommand{\bth}{\mbox{\boldmath$\theta$}}   %  bold theta in math mode
\newcommand{\dt}{\widetilde{d}}               %  d tilde in math mode
\newcommand{\m}{\mbox{\boldmath $m$}}         %  bold m in math mode
\newcommand{\M}{\mbox{\boldmath $M$}}         %  bold M in math mode
\newcommand{\mi}{{$-$}}                       %  large math minus sign
\newcommand{\n}{\mbox{\boldmath $n$}}         %  bold n in math mode
\newcommand{\ndot}{n{\rm\bf .}}               %  n sub-dot
\newcommand{\Nh}{\widehat{N}}                 %  N-hat
\newcommand{\nsp}{\hspace*{-.3in}}            %  large negative space
\newcommand{\p}{\phantom{5}}                  %  space for one digit
\newcommand{\pih}{\widehat{\pi}}              %  pi-hat
\newcommand{\pmi}{\phantom{$-$}}              %  space for wide minus
\newcommand{\ps}{\phantom{$*$}}               %  space for * footnote
\newcommand{\pc}{{\scriptsize$\bullet$}}      %  plotting dot character
\newcommand{\q}{\phantom{55}}                 %  space for two digits
\newcommand{\qs}{\hspace*{.1in}}              %  wide space
\newcommand{\re}{{\rm e}}                     %  Roman e (...to the power)
\newcommand{\bt}{\mbox{\boldmath $t$}}        %  bold t in math mode
\newcommand{\bT}{\mbox{\boldmath $T$}}        %  bold T in math mode
\newcommand{\U}{\mbox{\boldmath $u$}}         %  bold u in math mode
\newcommand{\X}{\mbox{\boldmath $X$}}         %  bold X in math mode
\newcommand{\x}{\mbox{\boldmath $x$}}         %  bold x in math mode
\newcommand{\Y}{\mbox{\boldmath $Y$}}         %  bold Y in math mode
\newcommand{\y}{\mbox{\boldmath $y$}}         %  bold y in math mode
\newcommand{\z}{\mbox{\boldmath $z$}}         %  bold z in math mode


% ------------------- my hyphenation rules ---------------------

\hyphenation{general-iza-tions}
\hyphenation{distri-bution}
\hyphenation{Zel-ter-man}

% ---------------  cover page  --------------------------------

\thispagestyle{empty}
\begin{center}
\vspace*{\fill}

{\Large\bf A Stopped Negative Binomial Distribution} \\ [1ex]

\vspace*{1.25in}
\begin{tabular}{l} 
 {\bf Michelle DeVeaux} \\
 {\bf Michael Kane${}^*$} \\
 {\bf Daniel Zelterman}  \\[.1in]

 Department of Biostatistics \\
 School of Epidemiology  \\
 \hspace*{.15in} and Public Health  \\
 Yale University       \\
 New Haven, CT  06520  \\[.2in]

\today
\end{tabular}
\end{center}

\vspace*{\fill}

\noindent${}^*$Email address for author: {\tt
michael.kane@yale.edu}.  
This research was supported by grants R01CA131301, R01CA157749, R01CA148996, 
R01CA168733, and PC50CA196530 awarded by the National Cancer Institute, and 
support from the Yale Comprehensive Cancer Center.

% ------------------------ Abstract ----------------------------

\newpage
\thispagestyle{empty}

\section*    {\bf   Abstract}

This paper introduces a new discrete distribution suggested by curtailed 
sampling rules common in early-stage clinical trials. We derive the 
distribution of the smallest number of independent Bernoulli($p$) trials 
needed in order to observe either $s$ successes or $t$ failures. The closed 
form expression for the distribution as well as the compound distribution are 
derived. Properties of the distribution are shown and discussed. A case study 
is presented showing how the distribution can be used to monitor sequential 
enrollment of clinical trials with binary outcomes as well as providing 
post-hoc analysis of completed trials.

\bigskip

\noindent{\bf Keywords:}  discrete distribution; curtailed sampling

% ---------------------------------------------------------------

\thispagestyle{empty}
\setcounter{page}{1}

% --------------------  1. Introduction   ----------------------
\section            {Introduction and Motivation}
% --------------------------------------------------------------


%In the Simon 2-stage Phase II clinical trial [\ref{Simon 1989}], a small 
%number of patients are enrolled onto a therapeutic 
%protocol for an initial stage. For each patient, we assess either success 
%or failure according to a 
%clinically useful criteria. Success might include tumor response to treatment
%or perhaps that the response is maintained for a minimum specified duration.
%In addition to failing to achieve the primary endpoint, treatment failure 
%might include adverse reaction to the drug or withdrawal due to toxicity. We 
%use a large number of these early successes as evidence for whether to treat
%additional patients in a second stage. A statistically small number of 
%successes in Stage I would lead us to terminate the trial without entering 
%Stage II.

Consider the design for a Phase II clinical trial to investigated the 
efficacy of iniparib in patients with breast cancer gene-associated (BRCA) 
ovarian cancer [\ref{Sanofi 2013}]. In the first stage, 12 patients were 
treated with iniparib (BSI-201/SAR240550). The endpoint was defined by 
Response Evaluation Criteria in Solid Tumor (RECIST). If at least two of
these patients respond favorably, then the trial will proceed to the next 
stage where additional patients are able to be treated. If fewer than two 
respond then the trial would be terminated. 

The maximum sample size is 12 but the number of patients necessary to 
reach and endpoint can be less. Our goal is to describe the distribution of 
the enrollment size of the design for 
planning purposes and to use this distribution to estimate the
success probability.  If all 12 patients are enrolled at once, 
as in the classic 
design, then the sample size is 12. However, in most clinical trials, the 
patients are enrolled sequentially, often with one patient's outcome realized 
before the next one enters the trial. In the present example, observing two 
successful patients allows us to proceed to the next stage. Similarly 11 
observed treatment failures also ends the stage. This sampling mechanism, in 
which the experiment ends as soon as any of the endpoints is reached, is 
call {\em curtailed sampling}. Under curtailed sampling the range of the 
sample size is between two and 12.

We assume each of the patient outcomes can be modeled as independent, 
identically distributed Bernoulli($p$) random variables. The realization of 
the trial is a sequence of these random variables that stops when either a 
specified number of success or failures is reached. As in the 
previous example, two successes were reached after enrolling 10 
patients (one in the third step and one at the $10^{th}$). The trial is 
visualized in Fig.~\ref{fig:kane_viz}. The vertical axis denotes the number of 
successful outcomes. The horizontal axis denotes the number of patients that 
have been enrolled. The horizontal and vertical line segments represent 
boundary endpoints for the trial. 

\begin{figure}[ht]
\includegraphics[width=\textwidth]{KanePlot.pdf}
\caption{
A hypothetical realization of the breast cancer trial.
}
\label{fig:kane_viz}
\end{figure}

%Our goals for this paper are three-fold. First, we derive the distribution for the number patients that need to be enrolled in the described sequential trial, which we refer to as the {\em Stopped Negative Binomial} (SNB) distribution. Using a sequential trial it is often possible to reduce the number of patients needed when compared to traditional design. Understanding of this distribution allows to create better, more {\em enrollment efficient} designs. Second, assuming a Beta distribution on the probability of success, we will derive the compound distribution of the SNB. Both the success probability and the SNB compound distribution can be updated as the individual outcome of patients are known and, as a result, better estimates of the distribution of patient-enrollment size can be derived as new data are received. Third, we will use the derived results to propose an enrollment design framework alternative to the Simon Two-stage Optimal trial design. We will show a hypothetical but representative case where the number of enrollments is less than those proposed by the Simon Two-stage and the uncertainty associated with both the individual success outcome as well as the trial is fully characterized.

The next section of this paper introduces our notation and basic results 
including the density of the distribution along with a description of 
it's relation to other distributions. Sections 2 and 3 derive the distribution 
based on a defined Bernoulli process and give some basic properties of the 
distribution. Section 4 derives the compound distribution using a Beta prior. 
Section 5 develops a post-hoc analysis of a completed trial. Section 6 is 
devoted to discussion and conclusions. The simulations, model fitting 
routines, and visualizations presented in this paper were generated using the 
snb package [\ref{Kane 2015}] for the R programming 
environment [\ref{R Core Team 2015}]

%Their individual outcomes are summarized as independent Bernoulli, %with either success according to a clinically useful criteria or else %as treatment failure. Failure might include adverse reaction to the %drug or withdrawal due to toxicity. Successes occur with probability %$\,p\; (0\leq p\leq 1)$.   

%More generally, suppose we have a sequence of independent Bernoulli$(p)\,$random variables $\,Z_1, Z_2, \ldots$. For positive integers $\,s\,$ and $\,t,$  we study the distribution of the smallest number $\,Y\,$ such that $\,Z_1, Z_2, \ldots, Z_Y\,$ %contains either $\,s\,$ successes of $\,t\,$ failures.

% ---------------------------  2. Notation ----------------------
\section   {Notation and Summary of Distributional Results}
\label{notation.section}
% --------------------------------------------------------------

Let $\,b_1, b_2, \ldots \,$ denote a sequence of independent, identically 
distributed, Bernoulli random variables with $\mathbb{P}\{b_i=1\}=p$, for 
probability parameter $0\leq p \leq 1$. In the clinical trial setting 
$\,b_i = 0$ corresponds to an unsuccessful outcome.  Let $s$ and $t$ be 
positive integers.  Define the stopped negative binomial (SNB) random 
variable $Y$ as the smallest 
integer value such that $\,\{b_1, \ldots , b_Y\}\,$ contains {\em either} 
$\,s\,$ successes {\em or} $\,t\,$ failures. The distribution of $\,Y\,$ has 
support on integer values in the range 
\begin{equation*}                                     %   (3)
     \min(s,t) \leq \; Y \;\leq s+t-1  \label{range.y.eq}
\end{equation*}
and it is distributed as
\begin{equation} \label{eqn:pmf}
\mathbb{P}[Y=k] = S(k, p, s)\left[s \leq k\right] + 
  S(k, 1-p, t) \left[ t \leq k \right]
\end{equation}
where 
\begin{equation} \label{eqn:N}
S(k, p, s) = {k-1 \choose s-1} p^s (1-p)^{k-s} 
\end{equation}
$S$ is the negative binomial probability truncated to a subset of the support 
- in this case, $s$ and $t$ respectively.
From Eqn.~\ref{eqn:pmf} one can see that the mass at 
any point on the support of the distribution may come from either 
$X_k = s$, captured in the function $S$, or $X_k = k-t$, captured in the 
function $T$. The support of 
$S$ is defined by $s \leq S \leq s+t-1$ and the support of $T$ is defined 
by $t \leq T \leq s+t-1$.

%It should be noted that
%this result expressed using the de Finetti where indicator functions and
%sets are equivalent. A more complete explanation of this equivalence can be 
%found in [\ref{Pollard 2002}]. 

% ------------   Section 3: Random Walk  --------------
%\section {Deriving the SNB with a Random Walk}
%\label{random.walk.section}
% --------------------------------------------------------------

To show the result in Eqn.~\ref{eqn:pmf} consider the 
process $\mathbf{X} = \left[X(k) : k \in \mathbb{Z}_{\ge 0} \right]$ 
with $X(0)=0$ and
\begin{equation*} \label{eqn:proc}
X_{k+1} = X_k + b_{k+1} \left[ k-t < X_k < s \right].
\end{equation*}
As before the $b_{k+1}$'s are distributed as Bernoulli($p$) for all $k$ in the 
domain of $\mathbf{X}$. The process can be conceptualized as a series of coin 
flips that stops when either $s$ heads or $t$ tails are reached. At each step 
a coin is flipped. If it is heads, the process advances one diagonally in the 
positive horizontal and vertical direction. Otherwise, it advances in the 
positive horizontal direction only. When the process stops
either $X_k = s$ or $X_k = k-t$. One example is given in 
Fig.~\ref{fig:kane_viz}.

\begin{prop}
The distribution of the stopping time $\argminl_k [X_k \geq s, X_k \leq k-t]$ 
is equal to the SNB and is given in (\ref{eqn:pmf}). 
\end{prop}
\begin{proof}
%The proof will proceed in two parts. First, a combinatorial justification 
%will be given for the probability mass value on each element of the support. 
%Second, it will be shown that the sum of the masses over the support sums to 
%one.

The probability that a given realization of $\mathbf{X}$ reaches $s$ at 
the $k$th outcome is the probability that, at time $k-1$ there are $s-1$ 
successful outcomes and $k-s$ unsuccessful outcomes multiplied by 
the probability of a 
success at time $k$ (\ref{eqn:N}). The probability $X_k = k-t$ 
is the probability that, at outcome $k-1$ there are $k-t$ successful outcomes 
and $t-1$ unsuccessful outcomes multiplied by the probability of an 
unsuccessful outcome at time $k$.  Finally, we need to show that 
the sum
\begin{equation} \label{eqn:sum_proof}
R = \sum_{k=s}^{s+t-1} {k-1 \choose s-1} p^s (1-p)^{k-s} + \sum_{k=t}^{s+t-1} {k-1 \choose k-t} p^{k-t} (1-p)^t
\end{equation}
is equal to one.
First, substitute $i=k-s$ in the first summation and
$j=k-t$ in the second. 
\begin{equation*}
{j+s-1 \choose s-1} = {j+s-1 \choose j}
\end{equation*}
Then %(\ref{eqn:sum_proof}) becomes:
%\begin{equation*}
%R = \sum_{i=0}^{t-1} {i+s-1 \choose s-1} p^s (1-p)^i +
%\sum_{j=0}^{s-1} {j+t-1 \choose j} p^j (1-p)^t.
%\end{equation*}
$R$ can be written
as the cumulative distribution function of two 
negative binomial distributions
\begin{equation} \label{eqn:transformed_sum}
R = \sum_{i=0}^{t-1} {i+s-1 \choose i} p^s (1-p)^i \; + \;
\sum_{j=0}^{s-1} {j+t-1 \choose j} p^j (1-p)^t.
\end{equation}

Let $I_p(s, t)$ be the {\em regularized incomplete beta function}, which is 
also one minus the c.d.f. of the negative binomial distribution. This
function satisfies $I_p(s, t) = 1-I_{1-p}(t, s)$.  Then it can be proven
that \ref{eqn:pmf} is a valid probability mass by showing that the
probability mass function sums to one.
\begin{align*}
R = \sum_{i=0}^{t-1} &{i+s-1 \choose i} p^s (1-p)^i +
\sum_{j=0}^{s-1}  {j+t-1 \choose j} p^j  (1-p)^t \\
   &= 1-I_p(s, t) + 1 - I_{1-p}(t, s) \\
   &= 1. 
\end{align*}
\end{proof}


\begin{prop} Let $S$ be distributed SNB with parameters $p$, $s$, and $t$.
Then the moment generating function (MGF) of $S$ is
\begin{equation} \label{eqn:mgf}
\mathbb{E}~e^{xS} = \frac{p e^x}{1 - qe^x} I_{qe^x} (s, t) + \frac{qe^x}{1-pe^x} I_{pe^x}(s, t)
\end{equation}
for $q = 1-p$ when $x \leq -\log(q)$ and $x \leq -\log(p)$.
\end{prop}
\begin{proof}
By definition, the MGF of the SNB is:
\begin{equation*}
\mathbb{E}~e^{xS} = \sum_{k=s}^{s+t-1} {k-1 \choose s-1} p^s q^{k-s} e^{kx} 
  + \sum_{k=t}^{s+t-1} {k-1 \choose k-t} p^{k-t} q^t e^{kx}
\end{equation*}
and this can be rewritten as:
\begin{equation} \label{eqn:first_sum}
\mathbb{E}~e^{xS} = \sum_{k=s}^{s+t-1}{k-1 \choose s-1} (pe)^{sx} (qe^x)^{k-s} 
  + \sum_{k=t}^{s+t-1}{k-1 \choose k-t} (qe^x)^t (pe^x)^{k-t}.
\end{equation}
Taking the first summation in Equation \ref{eqn:first_sum}:
\begin{align*}
\sum_{k=s}^{s+t-1}{k-1 \choose s-1} (pe)^{sx} (qe^x)^{k-s} &= 
  \left(\frac{pe^x}{1 - qe^x}\right)^s \ \ \sum_{k=s}^{s+t-1} {k-1 \choose s-1} 
    (qe^x)^{k-s} (1-qe^x)^s) \\
  &= \left(\frac{pe^x}{1 - qe^x}\right)^s I_{qe^x}(s, t).
\end{align*}
Since the incomplete beta function has support on zero to one 
$qe^x \leq 1$. This implies that $x \leq -\log(q)$.

A similar expression can be derived using the same calculation with 
the constraint that $x \leq -\log(p)$. The result
follows from the afore mentioned property of the regularized incomplete
beta function.
\end{proof}

\begin{figure}[p!]
\begin{center}
\includegraphics[width=\textwidth]{shapes.pdf}
\end{center}
\caption{Different shapes of the SNB distribution with parameters ($s$, $t$, $p$), as given. Red indicates mass contributed by hitting $s$, teal indicates
mass contributed by hitting $t$. \label{shapes.fig}}
\end{figure}

The probability mass function of $\,Y\,$ has a variety of shapes for different choices of the parameters $(s,\, t,\,p)$.
These shapes are illustrated in Fig.~\ref{shapes.fig}.
The SNB is related to the negative binomial distribution. Specifically, if 
$t$ is large then the $Y-s$ has a negative binomial distribution 
with
\begin{equation*}                                    %   (1)
   \mathbb{P}\{Y=s+j \}        \label{nb1.eq}
          = {{s+j-1}\choose{s-1}} p^s (1-p)^j
\end{equation*}
for $\,j=0, 1,\ldots\,$. A similar statement can be made when $s$ is large
and $t$ is small.

For the special case of $\,s=t,$ the distribution of $\,Y\,$ is the 
riff-shuffle, or minimum negative binomial distribution~[\ref{Uppuluri 1970}].
Similar derivations of the closely-related maximum negative binomial discrete 
distributions also appear in~[\ref{Zhang 2000}, \ref{Zelterman 2004}]. 
The maximum negative binomial is the smallest number of outcomes necessary
to observe at least $c$ successes {\em and} $c$ failures, but the SNB give
the number of coin flips to observe {\em either} $s$ heads or $t$ tails.

%\subsection{Updating the SNB distribution}

%Under the presented random walk construction if the process is observed before reaching one of its endpoints the conditional distribution is a reparametrized SNB distribution whose parameters are determined by $X_k$ and $k$, the process value and time step. For example, in the breast cancer trial the distribution of the number of patients is parameterized as $SNB(2, 11, p)$. At the third enrollment there is one successful outcome. Conditioned on this information, the number of additional patient enrollments is $SNB(1, 9, p)$. As new outcomes are collected the conditional distribution can be re-derived to get better estimates of the total patient enrollment. Furthermore, if our task is to estimate $p$ along with the patient enrollment distribution, this parameter can also be updated as new data are received.  

\section{A Bayesian Analysis}

%The previous section derives a new distribution for a sequential Bernoulli process where the process is stopped when either a total number of successes is reached or a total number of failure is reached. The process can be alternatively conceptualize as a series of coin flips that stops when either $s$ heads or $t$ tails are reached. At each step a coin is flipped. If it is heads then one step is taken in the positive direction on the vertical axis. Otherwise, a step is taken in the positive direction on the horizontal axis. The walk stops when either $s$ is reached on the vertical axis or $t$ is reached on the horizontal axis. A visual example is shown in Fig. \ref{fig:zelterman_viz}.

%\begin{figure}[ht]
%\includegraphics[width=\textwidth]{ZeltermanPlot.pdf}
%\caption{
%A realization of $\mathbb{X}$ with $p=0.3$, $s=3$, and $t=10$. An arrow to the right indicates a coin flip of ``tail'' and
%an arrow in the vertical direction indicates ``head.''
%}
%\label{fig:zelterman_viz}
%\end{figure}

Let us assume that $p$ has a Beta distribution, with constant prior 
parameters $\alpha$ and $\beta$. Then the closed form 
posterior distribution of the SNB is
\begin{align}
\mathbb{P} \left[Y = k | \mathbf{b}, \alpha, \beta \right] &= 
  {k-1 \choose s-1} \frac{B\left(\alpha+s, k-s+\beta \right)}{B(\alpha, \beta)} 
  [s \leq k \leq s+k-1] + \nonumber \\
  & {k-1 \choose k-t} 
  \frac{B\left(\alpha + k - t, t+\beta\right)}{B(\alpha, \beta)} 
  [t \leq k \leq s+k-1]
\end{align}
where $B$ is the Beta function and $\mathbf{b}$ is the sequence of Bernoulli data.

%When $p$ is known, the probability mass function (PMF) of the number of steps 
%until the process is stopped is given by (\ref{eqn:pmf}). However, many times 
%this is not the case. 
%To do this, we will assume that $p$'s probability in the absence of 
%available data is $\text{Beta}(1/2, 1/2)$, the Jeffrey's prior. This 
%distribution was chosen for two reasons. First, it is the conjugate prior 
%of the (Negative) Binomial distribution and updating the distribution as data 
%are received is straightforward. Second, it is objective and limits the number 
%of ``prior coin-flips'' to one. This means that data are more heavily 
%influence the distributional characteristics of $p$ with more data providing 
%more proportional influence.

\begin{prop}
The posterior PMF of the Stopped Negative Binomial distribution with a Beta($\alpha$, $\beta$) prior is:
\begin{align} \label{eqn:posterior}
f(k | s, t, \alpha, \beta) &= 
  {k-1 \choose s-1} \frac{B\left(\alpha+s, k-s+\beta \right)}{B(\alpha, \beta)}     [s \leq k \leq s+k-1] + \nonumber \\
  & {k-1 \choose k-t} 
    \frac{B\left(\alpha + k - t, t+\beta\right)}{B(\alpha, \beta)} 
    [t \leq k \leq s+k-1]
\end{align}
where $B$ is the Beta function.
\end{prop}
\begin{proof}
For notational simplicity, assume that $s,t \leq k \leq s+t-1$. When this is not the case appropriate terms should be removed as dictated by the indicator functions.
\begin{align*}
f(k | s, t, \alpha, \beta) = \frac{1}{B(\alpha, \beta)} & \int_0^1 {k-1 \choose s-1} p^{\alpha +s -1} \left(1-p\right)^{k-s+\beta-1} + \\
 & {k-1 \choose k-t} p^{k-t+\alpha-1}\left(1-p\right)^{t+b-1} dp \\
= \frac{1}{B(\alpha, \beta)}  {k-1 \choose s-1} & \int_0^1  p^{\alpha +s -1} \left(1-p\right)^{k-s+\beta-1} dp + \\
 & \frac{1}{B(\alpha, \beta)} {k-1 \choose k-t} \int_0^1  p^{k-t+\alpha-1}\left(1-p\right)^{t+b-1} dp
\end{align*}
The result follows by applying the definition of the Beta function to the integral terms.
\end{proof}

Eqn.~\ref{eqn:posterior} suggests an alternative construction for 
Phase II clinical trials.  Patients could be 
enrolled sequentially until there are enough patients to guarantee the 
desired level of power or until a specified number of adverse events are 
recorded. In cases such as these the endpoints $s$ and $t$ are known and 
new outcomes can be incorporated into the estimate of
$p$ to provide updated odds of successful enrollment. Later it will be shown 
that estimates of the distribution of $p$ can even be used after an endpoint 
has been reached to determine the amount of uncertainty in the endpoint. 

\section{A Bayesian Design for Clinical Trials}

%Returning to the breast cancer trial introduced in the first section, consider
%the first stage, having a sample size of 12. A
%response rate of 2/12 stage corresponds to an individual 
%probability of responding of $p=0.1667$. The standard deviation of this 
%probability is $\sqrt{p(1-p)/12}=0.1076$. If we were to use the Central 
%Limit Theorem then the 95\% confidence interval around the mean would 
%be $[-0.0442, 0.3776]$. While this approach is problematic since the number 
%of samples (12 in this case) is small it does indicate how much uncertainty 
%is built into the first stage. If we consider samples from both stages then 
%the standard deviation drops to 0.0630 and the 95\% confidence interval 
%would be $[0.0432, 0.2901]$.

The Simon Two-stage Optimal Design faces two problems. 
First, success in the first stage is not a good indicator of success 
in the second stage and at the same time failure in first stage in no way 
implies failure in the second stage, especially when $p$ is close to it's 
threshold value. The variance in the estimate is so large that it is a poor 
estimator. The second difficulty is that the design is not {\em enrollment 
efficient}. That is, if the first stage is successful then additional patients 
are enrolled. Enrollment, especially for certain types of cancer, can be very 
difficult and can take time to find eligible patients. Furthermore, if 
all 12 of the patients in the first stage respond positively, should we 
really need to enroll additional patients?

Properly quantifying uncertainty in the process implies quantifying 
uncertainty in $p$, which stems from two sources. First is the number of 
samples we receive. An estimate of $p$ will be based on data and generally 
the more data we have, the better our estimate. If the data are small, as 
they often are in clinical trials, the less reliable our estimate is. The 
second source comes from heterogeneity in the data. While models may assume 
that the probability of success $p$ is a single value this is often not the 
case in real-world data. In clinical trials for example an individual's chance 
of having an adverse outcome may be influenced by any number of underlying 
factors, which may vary across individual. In these cases $p$ may not be a 
single numeric value and may be better conceptualized as a realization over 
an entire distribution. A frequentist approach will not capture this variation.

We would like to be able to estimate $p$ and capture both sources of uncertainty. For cases where the process has not yet reached an endpoint, we would like to be able to provide updated estimates of $p$, its uncertainty, and the probabilities of hitting each endpoint. If the process has reached an endpoint, we would again like a estimate of $p$ and its uncertainty along with the probability that another endpoint could have been reached, given the data. 

\subsection{Quantifying Uncertainty in the Simon Two-stage Optimal Design}

Both of the problems in the Simon Two-stage Optimal Design stem from the fact that it fails to quantify the uncertainty in the estimate of the success probability. A framework based on the SNB distribution addresses this failure and 
provides the added features of monitoring the trial as new results are 
received, providing estimates of how many new enrollees are needed to complete 
a trial, and quantifying the uncertainty in the success probability.

Let's start by recasting the breast cancer study in the context of the SNB 
framework, assuming sequential enrollment. The first stage will stop when 
either two positive responses are reached or 11 negative responses. This 
corresponds to a SNB distribution with $s=2$ and $t=11$. Next, let's assume 
that, in the first stage, there were two successes. A second success could 
occur during the second, through a tenth enrollment. At each of these times we 
would like to know the estimate for $p$ along with the lower five percent 
quantile of the distribution. 

\begin{figure}[ht]
\includegraphics[width=\textwidth]{uncertainty.pdf}
\caption{
A visualization of the mode and the 5\% quantile of the estimate of the distribution of $p$.
}
\label{fig:simon}
\end{figure}

To better understand how the distributional estimate of $p$ changes in the 
number of steps needed for two successes data were simulated for each 
potential stopping point from two to 11. That is, we will fit the
Beta distribution with the Jeffrey's prior to the sequence of coin flips 
$\{[1,1], [1, 0, 1], [1, 0, 0, 1], ... \}$ in the support of the SNB with
$s=2$ and $t=11$.  Fig. \ref{fig:simon} shows the 
change in the mode of the beta distribution, which is also the Maximum 
Likelihood Estimator (MLE) as well as the five percent quantile of the 
distribution.  The graph shows, for example, that if a second success is 
found with the third enrollment then the MLE estimate of $p$ is 0.5 and with 
95\% confidence $p$ is at least 0.25. In this case investigators could have 
moved on to the next stage of the trial with only three enrollments as 
opposed to 12.

\begin{table}
\begin{center}
\begin{tabular}{|c|c|c|} \hline
{\bf Steps to 2 successes} & {\bf Mode of $p$} & {\bf 5\% quantile} \\ \hline
2 & 0.7500  & 0.4307 \\ \hline
3 & 0.5000  & 0.2355 \\ \hline
4 & 0.3000  & 0.1278 \\ \hline
5 & 0.1875  & 0.0763 \\ \hline
6 & 0.1250  & 0.0497 \\ \hline
7 & 0.0882  & 0.0347 \\ \hline
8 & 0.0652  & 0.0254 \\ \hline
9 & 0.0500  & 0.0194 \\ \hline
10 & 0.0398 & 0.0153 \\ \hline
11 & 0.0319 & 0.0123 \\ \hline
\end{tabular} 
\end{center}
\caption{
A table of the mode and the 5\% quantile of the estimate of the distribution of $p$.
}
\label{tab:simon}
\end{table} 

Table \ref{tab:simon} shows the actual values used to generate Fig. 
\ref{fig:simon}. 
%In the beginning of this section it was shown that, at the 
%end of stage two, the breast cancer study actually tests whether the 
%probability of success is greater than 0.0432 with 95\% confidence and 
%requires 35 patients to do so. With the Stopped Negative Binomial model we 
%only need six patients for the same level of confidence. This represents a 
%tremendous amount of savings in time, effort and resources.

\subsection{Trial Monitoring}

Like most Bayesian Adaptive Clinical trials the SNB approach allows for monitoring of trial and can be updated as new results are received without losing 
power. This is because the object under investigation is not a point estimate 
of the success probability; it is the uncertainty in the distribution of the 
success probability. Point estimates of the success probability can be 
regarded as artifacts of our parameterization of the underlying distribution. 
Every point estimate calculated is accompanied by the quantification of the 
uncertainty around it. This estimate and its uncertainty can be calculated at 
any time and may be useful in evaluating ongoing trials as well as providing a 
post-hoc evaluation of trials that have been completed.

\begin{figure}[ht]
\includegraphics[width=\textwidth]{hypo_traj.pdf}
\caption{
A visualization of the mode and the 5\% quantile of the estimate of the distribution of $p$ for a hypothetical sequence of patient enrollments.
}
\label{fig:hypo}
\end{figure}

Consider the trial described above. Suppose eight patients were enrolled and 
there was a single success on the third trial (visualized in Fig. 
\ref{fig:hypo}). The mode of the beta distribution can only be calculated when 
there is at least one success. The mean of the distribution can be
taken otherwise. At the third time step there is an increase in the estimate and the 5\% quantile corresponding to the success. 

\begin{figure}[ht]
\includegraphics[width=\textwidth]{conditional_snb.pdf}
\caption{
The conditional Stopped Negative Binomial distribution after eight patients with one success.
}
\label{fig:conditional_snb}
\end{figure}

Fig \ref{fig:hypo} shows that at the eighth enrollment the estimate of $p$ is 
about 0.19 and, from the lower 5\% quantile, we can see that there is
a 95\% probability that $p$ is greater than 2.5\%. 
Fig. \ref{fig:conditional_snb} shows the posterior SNB distribution for this 
hypothetical trial at the eighth step.  Successes may occur 
at steps nine through 11 with probability 26.7\%. The remaining probability
mass corresponds to reaching the $11^{th}$ step without a second success.

\subsection{Post-hoc Evaluation}

Suppose that, in our example trial, a second successful outcome was reached 
on the $10^{th}$ patient. From the data collected we have shown how to create 
an estimate for the distribution of $p$. Based on that estimate we may like to 
see the probability of a different outcome. We may want to answer the
question, ``What is the probability that we don't get two successful outcomes?''
We may also ask what is the distribution for the number of patients 
enrolled for a similar study?

If there are two successful enrollments with the second one occurring 
on the $10^{th}$ trial the success probability $p$ is distributed as 
$Beta(2.5, 8.5)$ (assuming the Jeffrey's prior). 
Using the mode, a point estimate yields $\hat{p} = 0.1667$. The standard 
deviation of the estimate is 0.1210. The probability of two successes during 
the trial is 0.6843 and the probability of failure is 0.3157. The distribution 
of the stopping times, along with contributions from two positive events and 
less than two positive events is shown below.

\begin{figure}[ht]
\includegraphics[width=\textwidth]{post_hoc.pdf}
\caption{
The post-hoc Stopped Negative Binomial distribution after the trial ends with 10 patient enrollments and two successes.
}
\label{fig:post_hoc}
\end{figure}

\section{Discussion and Conclusion}

We have presented a new discrete distribution by curtailed sampling rules common in early-stage clinical trials, which we refer to as the Stopped Negative Binomial distribution. The distribution models the stopping time of a sequential trial where the trial is stopped when a number of events are accumulated. The 
posterior distribution was derived for the case when the event probability $p$ 
has a Beta distribution. Using a trial description from 
\url{clinicaltrials.gov} we showed how the SNB is an integral part of trial 
monitoring, and post-hoc analysis and can be used in a framework alternative 
to the Simon two-stage optimal design while providing better estimates of the 
event probability along with quantifying the uncertainty associated with the 
estimates. As a result, we were able to show fewer patient enrollments and 
achieve the same estimates, when added in a hypothetical, but representative, 
clinical trial.

Current work focuses on the generalization of the distribution and the application in other areas of clinical trials. Adverse outcomes in particular are another area that could greatly benefit both from the presented distribution and its framework. For these trials monitoring may need to be performed not only on the outcome of an intervention but also on the safety of the trial. This especially 
true in areas like late-stage cancer treatments where the treatment is harsh 
and patients may be forced to drop out as a result of the side-effects of the 
treatment; a matter independent of the outcome. Other application areas 
include providing designs that allows clinicians to balance uncertainty and 
success probability with a minimum number of patient enrollees.

%   ------------------------    References     -----------------------


% -------------------------  References ---------------------------
\section*     {\bf References}
% -----------------------------------------------------------------


\begin{enumerate}

\item[\bibref{Kane 2015}]
Kane MJ. (2015). snb: The Stopped Negative Binomial. R package version 0.1. 2015. Available from \url{https://github.com/kaneplusplus/snb}

%\item[\bibref{Pollard 2002}]
%Pollard D. {\it A User's Guide to Measure Theoretic Probability}, 
%Cambridge University Press, 2002.

\item[\bibref{R Core Team 2015}]
R Core Team (2015). R: A language and environment for statistical
computing. R Foundation for Statistical Computing, Vienna, Austria. 2015. URL \url{http://www.R-project.org/}.

\item[\bibref{Sanofi 2013}]
Sanofi Pharmaceutical Company. A Phase 2, Single Arm Study of BSI-201 in Patients With BRCA-1 or BRCA-2 Associated Advanced Epithelial Ovarian, Fallopian Tube, or Primary Peritoneal Cancer. In: ClinicalTrials.gov [Internet]. Bethesda (MD): National Library of Medicine (US). 2000- [cited 2015 March 27]. Available from: \url{https://clinicaltrials.gov/ct2/show/study/NCT00677079} Identifier: NCT00677079.

\item[\bibref{Simon 1989}]
Simon R.  Optimal two-stage designs for Phase II clinical trials. {\it Controlled Clinical Trials\/} 1989; {\bf 10}: 1--10.

\item[\bibref{Uppuluri 1970}] 
Uppuluri VRR, Blot WJ. ``A probability distribution arising in a riff-shuffle.'' {\it Random Counts  in Scientific Work, 1: Random Counts in Models and Structures}, 1970. G.P. Patil (editor), University Park: Pennsylvania State University Press, pp  23--46.

\item[\bibref{Zhang 2000}]
Zhang Z, Burtness BA, Zelterman D.  The maximum negative binomial distribution. {\em Journal of Statistical Planning and Inference\/}  2000; {\bf 87}: 1--19.

\item[\bibref{Zelterman 2004}]
D Zelterman. {\it Discrete Distributions: Application in 
the Health Sciences}, New York: J. Wiley. 2004. xix + 277pp

\end{enumerate}

% ------------------------------------------------------------------




% ------------------------------------------------------------------
% --------------------   Tables and Figures   ----------------------
% ------------------------------------------------------------------




% -------------------------------------------------------------------


% -------------------------------------------------------------------
% ------------------------  FIGURES  --------------------------------
% -------------------------------------------------------------------

%\newpage 
% -------------------------------------------------------------------


%\newpage     %  Figure 1: functional forms of the mass function  


% -------------------------------------------------------------------

\end{document}

% ------------------  end of this file --------------------------



